\DeclareFontFamily{U}{mathb}{\hyphenchar\font45}
\DeclareFontShape{U}{mathb}{m}{n}%
{<-6> mathb5 %
 <6-7> mathb6
 <7-8> mathb7 %
 <8-9> mathb8 %
 <9-10> mathb9 %
 <10-12> mathb10 %
 <12-> mathb12 }%
 {}
\DeclareSymbolFont{mathb}{U}{mathb}{m}{n}
\DeclareMathSymbol{\sqbullet}{\mathbin}{mathb}{"0D}
\DeclareMathSymbol{\sqbollet}{\mathbin}{mathb}{"05}

%%%%%%%%%%%
% Faltung %
%%%%%%%%%%%
\newcommand{ \conv }{ \mathbin{*} }
\newcommand{ \tconv }{ \mathop{\textstyle\bigasterisk} }
\newcommand{ \dconv }{ \mathop{\bigasterisk} }

%%%%%%%%%%%%%%%%%%%%%%%
% Potenzierte Faltung %
%%%%%%%%%%%%%%%%%%%%%%%
%\newcommand{ \pconv }[ 1 ]{ \mathbin{\mathop{*}\limits_{#1}} }
%\newcommand{ \pconv }[ 1 ]{ \mathbin{*_{#1}} }
%\newcommand{ \pconv }[ 1 ]{ \mathbin{\overset{#1}{*}} }
\newcommand{ \pconv }[ 1 ]{ \mathbin{*^{#1}} }


%%%%%%%%%%%%%%%%%%%
% Supremalfaltung %
%%%%%%%%%%%%%%%%%%%
\newcommand{ \sconv }{ \mathbin{\sqbullet} }

\newcommand{ \oldsconv }{ \mathbin{\triangle\hspace*{-1.39ex}\cdot\hspace*{.95ex}} }% supremale Faltung
\newcommand{ \oldiconv }{ \mathbin{\raisebox{.4ex}{\rotatebox[origin=c]{180}{$\scnv$}}} }% infimale Faltung
\newcommand{ \oldsldconv }[ 2 ]{ #1\mathbin{\hspace*{-0.35ex}\,{\triangle\hspace*{-1.6ex}\backslash\hspace*{1.1ex}}\hspace*{-0.35ex}}#2 }% supremale linksseitige Entfaltung
\newcommand{ \oldsrdconv }[ 2 ]{ #1\mathbin{\hspace*{-0.35ex}\,{\triangle\hspace*{-1.5ex}/\hspace*{1ex}}\hspace*{-0.35ex}}#2 }% supremale rechtsseitige Entfaltung

\newcommand{ \cdiv }{ \mathbin{ \mathrlap{ / } \conv } }
\newcommand{ \scdiv }{ \mathbin{ \mathrlap{ / } \sconv } }


\newcommand{ \scrdiv }{ \mathop{\mathrlap{/ }{ \sconv}} }
\newcommand{ \crdiv }{ \mathop{\mathrlap{/ }{ \conv}} }

\newcommand{ \scrdivLsc }{ \mathop{\opLscH{\mathrlap{/ }{ \sconv}}} }


\newcommand{ \sconvUsc }{ \mathbin{\opUscH{\sqbullet}} }

%%%%%%%%%%%%%%%%%%
% Infimalfaltung %
%%%%%%%%%%%%%%%%%%
\newcommand{ \iconv }{ \mathbin{\sqbollet} }



%%%%%%%%%%%%%%%%%%%%%%%%
% Kegelideal Versionen %
%%%%%%%%%%%%%%%%%%%%%%%%
%\newcommand{ \convCI }{ \overset{ \joiny }{ \conv } }
%\newcommand{ \convCI }{ \mathbin{ \widehat{ \conv } } }
\newcommand{ \convCI }{ \mathbin{ \widecheck{ \conv } } }
\newcommand{ \convLS }{ \mathbin{ \conv_{ \ssLS } } }
%\newcommand{ \sconvCI }{ \sconv^{\joiny}}
%\newcommand{ \convCI }{ \mathbin{\mathop{\conv}\nolimits_{\sssfont{ I }}} }
\newcommand{ \sconvCI }{ \mathbin{\mathop{\sconv}\nolimits_{\sssfont{ I }}} }

%\newcommand{ \convCIP }{ \mathbin{\mathop{\conv}\nolimits_{\sssfont{ I },\mathrm{ P }}} }
%\newcommand{ \convCIP }{ \mathbin{\mathop{\widetilde{ \conv }}\nolimits_{\sssfont{ I }}} }
\newcommand{ \convCIP }{ \mathbin{\widetilde{ \conv }} }

\newcommand{ \sconvCIUsc }{ \mathbin{\mathop{\opUscH{\sqbullet}}\nolimits_{\sssfont{ I }}} }

\newcommand{ \cdivCI }{ \mathbin{ \mathop{ \cdiv }\nolimits_{ \sssfont{ I } } } }

\newcommand{ \cdivCIT }{ \mathbin{ \mathop{ \cdiv }\nolimits_{ \mathrm{ T } } } }

\newcommand{ \cdivCIP }{ \mathbin{ \mathop{ \mathbin{ \mathrlap{ / } \widetilde{\conv} } } } }


\newcommand{ \PV }{ \mathrm{PV} }



\newcommand{ \opd }{ \mathrm{ d } }
\newcommand{ \intd }[ 1 ]{ \,\opd{ #1 } }
%\newcommand{ \diff }{ \mathrm{d}}
\newcommand{ \tint }{ { \textstyle\int } }


\newcommand{ \rska }[ 2 ]{ \ska{ #1 }{ #2 }^\vee }
\newcommand{ \rTska }[ 3 ]{ \ska{ #1 }{ #2 }^\vee_{ #3 } }

\newcommand{ \convPol }[ 1 ]{ \left( #1 \right)^{ \conv 1 } }
\newcommand{ \convPolct }[ 2 ]{ \left( #1 \right)^{ \conv 1 }_{ #2 } }

\newcommand{ \convPolss }[ 1 ]{ \left( #1 \right)^{ \conv 1 } }
\newcommand{ \convPolssct }[ 1 ]{ \left( #1 \right)^{ \conv 1 }_{ \mathrm{ T } } }

%\newcommand{ \convPolTop }[ 1 ]{ { #1 }^{ \conv 1 } }
%\newcommand{ \convPolTopct }[ 1 ]{ { #1 }^{ \conv 1 }_{ \mathrm{ T } } }

%\newcommand{ \convPolBor }[ 1 ]{ { #1 }^{ \conv 1 } }
%\newcommand{ \convPolBorct }[ 1 ]{ { #1 }^{ \conv 1 }_{ \mathrm{ T } } }


\newcommand{ \opCod }[ 2 ]{ { #1 }^{ \Delta #2 } }
%\newcommand{ \tens }{ \otimes }



\newcommand{ \opTIl }[ 1 ]{ {#1}{}_{ \mathrm{ T } } }
\newcommand{ \opCD }[ 1 ]{ {#1}{}^* }


\newcommand{ \sfCodmsrat }[ 2 ]{ \opCod{ \delta }{ #1 }_{ #2 } }
\newcommand{ \sfCodmsr }[ 1 ]{ \sfCodmsrat{ #1 }{} }



\newcommand{ \sfdeo }{ \tde }
\newcommand{ \sfde  }[ 1 ]{ \sfdeo_{ #1 } }

%\newcommand{ \sfDrcSeq }[ 1 ]{ \tDe( #1 ) }
%\newcommand{ \SfDrcSeq }[ 1 ]{ \tDe\left( #1 \right) }

\newcommand{ \sfDe }[ 1 ]{ \tDe( #1 ) }
\newcommand{ \SfDe }[ 1 ]{ \tDe\left( #1 \right) }



%\newcommand{ \convfrac }[ 2 ]{ \mathrlap{\conv}{-}\hspace{-2pt}\frac{ #1 }{ #2 } }
%\newcommand{ \convfrac }[ 2 ]{ { \mathop{ \,{\mathrlap{\conv}\hspace{-2pt}\frac{ \hphantom{\conv} #1 }{ \hphantom{\conv} #2 } } } } }
\newcommand{ \convfrac }[ 2 ]{ { \,{\mathrlap{\conv}\hspace{-2pt}\frac{ \hphantom{\conv} #1 }{ \hphantom{\conv} #2 } } } }


\newcommand{ \clConv }[ 1 ]{ \brClo{ #1 }_* }


\newcommand{ \clCCA }[ 1 ]{ \brClo{ #1 }_{\nrC,+,*} }