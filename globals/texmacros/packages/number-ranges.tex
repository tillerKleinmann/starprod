%%%%%%%%%%%%%%%%%%%%%%%%%%%%%%%%%
% Schriftart für Zahlenbereiche %
%%%%%%%%%%%%%%%%%%%%%%%%%%%%%%%%%
\newcommand{ \fontNR }[ 1 ]{ \mathbb{ #1 } }



%%%%%%%%%%%%%%%%%%%%%%%%%%%%%
% klassische Zahlenbereiche %
%%%%%%%%%%%%%%%%%%%%%%%%%%%%%
\newcommand{ \nrN }{ \fontNR{N} }
\newcommand{ \nrZ }{ \fontNR{Z} }
\newcommand{ \nrQ }{ \fontNR{Q} }
\newcommand{ \nrR }{ \fontNR{R} }
\newcommand{ \nrC }{ \fontNR{C} }

\newcommand{ \nrNn }{ \nrN_0 }
\newcommand{ \nrRp }{ \nrR_+ }
\newcommand{ \nrRpn }{ \nrR_{0+} }

\newcommand{ \nrZi }{ \nrZ^\times }
\newcommand{ \nrQi }{ \nrQ^\times }
\newcommand{ \nrRi }{ \nrR^\times }
\newcommand{ \nrCi }{ \nrC^\times }


\newcommand{ \nrRpi }{ \nrR_* }



%%%%%%%%%%%%%%%%%%%%%%%%%%%%%
% erweiterte Zahlenbereiche %
%%%%%%%%%%%%%%%%%%%%%%%%%%%%%
\newcommand{ \nrRe  }{ \overline{ \fontNR{R} } }
\newcommand{ \nrRep }{ \nrRe_+ }



%%%%%%%%%%%%%%%%%%%%%%%%%%%%%%%%%%%%%
% Spezielle komplexe Zahlenbereiche %
%%%%%%%%%%%%%%%%%%%%%%%%%%%%%%%%%%%%%
\newcommand{ \nrD }{ \fontNR{D} }    % abgeschlossene Einheitskreisscheibe
\newcommand{ \nrT }{ \fontNR{T} }    % komplexe Einheiten ( absolutbetrag = 1 )
\newcommand{ \nrL }{ \fontNR{L} }    % Riemannschei Fläche des Logarithmus
\newcommand{ \nrH }{ \fontNR{H} }    % rechte offene Halbebene



\newcommand{ \imu }{ \mathrm{i} }
\newcommand{ \enr }{ \mathrm{e} }


%%%%%%%%%%%%%%%%%%%%
% Spezielle Zahlen %
%%%%%%%%%%%%%%%%%%%%
\newcommand{ \infi }{ \infty }



%%%%%%%%%%%%%%%%%%%%%%
% Binäre Operationen %
%%%%%%%%%%%%%%%%%%%%%%
\newcommand{ \mult }{ \cdot }


%%%%%%%%%%%%%%%%%%%%%
% Unäre Operationen %
%%%%%%%%%%%%%%%%%%%%%
\newcommand{ \ceil }[ 1 ]{ \lceil #1 \rceil }
\newcommand{ \floor }[ 1 ]{ \lfloor #1 \rfloor }
\newcommand{ \abs }[ 1 ]{ \lvert #1 \rvert }
\newcommand{ \absv }{ { \abs{ \cdot } } }

\newcommand{ \Abs }[ 1 ]{ \left\lvert #1 \right\rvert }



\newcommand{ \itv }[ 2 ]{ \interval{#1}{#2} }
\newcommand{ \ritv }[ 2 ]{ \rinterval{#1}{#2} }
\newcommand{ \litv }[ 2 ]{ \linterval{#1}{#2} }
\newcommand{ \oitv }[ 2 ]{ \ointerval{#1}{#2} }