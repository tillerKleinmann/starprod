\section{On Definitions for the Star Product}


The relation between a continuous linear operator
$ L \colon \fsTest \to \fsDis $ and its kernel $ K_L \in \fsDis \ctepro \fsDis $
is explained by
\begin{align}
    \ska{ L(\ph) }{ \ps }
    =
    \ska{ K_L }{ \ph \tepro \ps }
    &&&\text{for all $ \ph,\ps \in \fsTest $.}
\end{align}
Or, in converse notation, the relation between a distribution
$ K \in \fsDis \ctepro \fsDis $ and the associated
continuous linear operator $ L_K \colon \fsTest \to \fsDis $
is explained by
\begin{align}
    \ska{ L_K( \ph ) }{ \ps }
    =
    \ska{ K }{ \ph \tepro \ps }
    &&&\text{for all $ \ph,\ps \in \fsTest $.}
\end{align}
This explains the star product $ K \star \ph $ via
\begin{align}
    \ska{ K \star \ph }{ \ps }
    :=
    \ska{ K }{ \ph \tepro \ps }
    &&&\text{for all $ \ph,\ps \in \fsTest $.}
\end{align}

The star product satisfies
\begin{align}
    ( \ps * ( K \star \ph ) )( x )
    =
    \int ( ( \ps \tepro \sfdeo ) * ( K \star \ph ) )(x,y) \ph(y) \intd{y}
    &&&\text{for all $ \ph,\ps \in \fsTest $, $ x \in \nrR $.}
\end{align}

In order to prove, that the star product of kernels
equals the composition of operators,
one has to prove that
\begin{align}
    \ska{ L_T( L_S( \ph ) ) }{ \ps }
    =
    \ska{ L_{T\star S} }{ \ph \tepro \ps }
    &&&\text{for all $ \ph,\ps \in \fsTest $.}
\end{align}


Let $ T \in \fsSmo \ctepro \fsDis = \fsSmo_x(\fsDis) $,
$ S \in \fsDis \ctepro \fsSmo = \fsSmo_y(\fsDis) $
and $ f,g \in \fsSmo(\nrR^2) $ such that
\begin{subequations}
    \begin{align}
        T(x,\cdot)g(\cdot,y) &\in \fsDisint
        \\
        f(x,\cdot)S(\cdot,y) &\in \fsDisint
    \end{align}
\end{subequations}
for all $ x,y \in \nrR $.
Then, one defines the star product $ f \star S$ and $ T \star g $
via distributional integration as
\begin{subequations}
    \begin{align}
        ( T \star g )( x,y )
        &:=
        \int T( x,\cdot ) g( \cdot, y )
        \\
        ( f \star S )( x,y )
        &:=
        \int f( x,\cdot ) S( \cdot, y )
    \end{align}
\end{subequations}
for all $ x,y \in \nrR $.
\begin{align}
    ( f \star g )(x,y)
    :=
    \int f(x,z) g(z,y) \intd{z}
\end{align}
for all $ x,y \in \nrR $.


Let $ T \in \fsDis \ctepro \fsSmo $ and $ S \in \fsSmo \ctepro \fsDis $.
Then the {\em star product $ T \star S $ of $ T $ and $ S $}
can be defined implicitly as
\begin{align}
    ( T \star S ) * ( \ph \tepro \ps )
    :=
    ( T * ( \ph \tepro \sfdeo ) ) \star ( S * ( \sfdeo \tepro \ps ) )
\end{align}
for all $ \ph,\ps \in \fsTest $,
using the star product of smooth functions whenever it exists.

Let $ T,S \in \fsDis \ctepro \fsSmo $.
Then the star product can be defined as
\begin{subequations}
    \begin{align}
        ( T \star S ) * ( \ph \tepro \sfdeo )
        &:=
        ( T * ( \ph \tepro \sfdeo ) ) \star S
    \end{align}
    whenever the distributional integral exists.
Let $ T,S \in \fsSmo \ctepro \fsDis $.
Then the star product can be defined as
    \begin{align}
        ( T \star S ) * ( \sfdeo \tepro \ps )
        &:=
        T \star ( S * ( \sfdeo \tepro \ps ) )
    \end{align}
\end{subequations}
for all $ \ph, \ps \in \fsTest $.


\begin{lemma}
    \label{lem:unismo}
    A subset $ B \subs \fsSmo $ is uniformly smooth
    if and only if $ B \in \ssBnd(\fsSmo) $.
\end{lemma}

\begin{lemma}
    Let $ T \in \fsDis \ctepro \fsDis $.
    Then $ T \in \fsDis \ctepro \fsSmo $ holds if and only if
    \begin{align}
        \brmSet{ T * ( \ph \tepro \sfdeo ) }{ \ph \in \Ph }
        \in
        \ssBnd( \fsSmo \ctepro \fsSmo )
        &&&\text{for all $ \Ph \in \ssBnd(\fsTest) $.}
    \end{align}
\end{lemma}

\begin{proof}
    With the notation
    \begin{align}
        F_\ph
        :=
        T * ( \ph \tepro \sfdeo )
    \end{align}
    it is clear, that
    \begin{align}
        \Abs{ \Ska{ ( \tDe^{0,n}_h T )( \cdot, y )  -
        ( \partial^{0,n}T )( \cdot, y ) }{ \ph } }
        =
        \Abs{ ( \tDe^{0,n}_h F_{\refl{\ph}} )(0,y)  -
        ( \partial^{0,n} F_{\refl{\ph}} )(0,y) }
    \end{align}
    for all $ n \in \nrN_0 $, $ h > 0 $, $ y \in \nrR $ and $ \ph \in \fsTest $.
    This proves the equivalence stated in this Lemma
    due to Lemma~\ref{lem:unismo}.
\end{proof}

\begin{theorem}
    Let $ T,S \in \fsDis \ctepro \fsSmo $.
    Then the star product can be defined as
    \begin{align}
        ( T \star S ) * ( \ph \tepro \sfdeo )
        :=
        ( T * ( \ph \tepro \sfdeo ) ) \star S
        &&&\text{for all $ \ph \in \fsTest $,}
    \end{align}
    whenever the distributional integral exists.
\end{theorem}
\begin{proof}
    Let $ \Ph \in \ssBnd(\fsTest) $.
    It has to be proved, that
    \begin{align}
        \BrmSet{ ( T * ( \ph \tepro \sfdeo ) ) \star S }{ \ph \in \Ph }
        \in
        \ssBnd\left( \fsSmo \ctepro \fsSmo \right)
        .
    \end{align}
    First, note that it is true in general, that
    \begin{align}
        \BrmSet{ (x,y) \mto ( T * ( \ph \tepro \sfdeo ) )(x,\cdot) \mult S(\cdot,y) }
        { \ph \in \Ph }
        \in
        \ssBnd\left( ( \fsSmo \ctepro \fsSmo )( \fsDis ) \right)
        ,
    \end{align}
    because the set $ \BrmSet{ x \mto ( T * ( \ph \tepro \sfdeo ) )(x,\cdot) }{ \ph \in \Ph } $
    belongs to $ \ssBnd( \fsSmo( \fsSmo ) ) $,
    the mapping $ y \mto S(\cdot,y) $ belongs to $ \fsSmo( \fsDis ) $ and
    multiplication is hypocontinuous as a mapping $ \fsSmo \times \fsDis \to \fsDis $.
    It would be sufficient to prove, that
    \begin{align}
        \BrmSet{ (x,y) \mto ( T * ( \ph \tepro \sfdeo ) )(x,\cdot) \mult S(\cdot,y) }
        { \ph \in \Ph }
        \in
        \ssBnd\left( ( \fsSmo \ctepro \fsSmo )( \fsDisint ) \right)
        .
    \end{align}
\end{proof}


\begin{theorem}
    \label{thm:spcau}
    Let $ T,S \in \fsDisaca \ctepro \fsSmocau $.
    Then the {\em star product $ T \star S $ of $ T $ and $ S $}
    belongs to $ \fsDisaca \ctepro \fsSmocau $
    and is well-defined by the formula
    \begin{align}
        ( T \star S ) * ( \ph \tepro \sfdeo )
        :=
        ( T * ( \ph \tepro \sfdeo ) ) \star S
        &&&\text{for all $ \ph \in \fsTest $.}
        \label{thm:spcau:eq}
    \end{align}
\end{theorem}
\begin{proof}
    Let $ \Ph \in \ssBnd(\fsTest) $.
    It has to be proved, that
    \begin{align}
        \BrmSet{ ( T * ( \ph \tepro \sfdeo ) ) \star S }{ \ph \in \Ph }
        \in
        \ssBnd\left( \fsSmo \ctepro \fsSmo \right)
        .
        \label{prf:thm:spcau:eq:1}
    \end{align}
    First, note that it is true, that
    \begin{align}
        \BrmSet{ (x,y) \mto ( T * ( \ph \tepro \sfdeo ) )(x,\cdot) \mult S(\cdot,y) }
        { \ph \in \Ph }
        \in
        \ssBnd\left( ( \fsSmo \ctepro \fsSmo )( \fsDis ) \right)
        ,
        \label{prf:thm:spcau:eq:2}
    \end{align}
    because
    $ \BrmSet{ x \mto ( T * ( \ph \tepro \sfdeo ) )(x,\cdot) }{ \ph \in \Ph } $
    belongs to $ \ssBnd( \fsSmo( \fsSmo ) ) $,
    the mapping $ y \mto S(\cdot,y) $ belongs to $ \fsSmo( \fsDis ) $ and
    multiplication is hypocontinuous
    as a mapping $ \fsSmo \times \fsDis \to \fsDis $.

    The supports of the functions in Equation \eqref{prf:thm:spcau:eq:2}
    are contained in a fixed bounded set
    if $ x,y $ are from a bounded set.
    Thus, the set on the left hand side
    in Equation \eqref{prf:thm:spcau:eq:2} is contained
    in $ \ssBnd\left( ( \fsSmo \ctepro \fsSmo )( \fsDiscpt ) \right) $,
    and thus, in $ \ssBnd\left( ( \fsSmo \ctepro \fsSmo )( \fsDisint ) \right) $.
    This inclusion implies Equation \eqref{prf:thm:spcau:eq:1},
    and thus, that $ T \star S \in \fsDis \ctepro \fsSmo $.
    Considering the supports, it is also clear,
    that $ T \star S \in \fsDisaca \ctepro \fsSmocau $.
\end{proof}