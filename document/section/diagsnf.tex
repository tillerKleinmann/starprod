\section{Diagonal Supnorm Functionals}
\label{dsnf}


Let $ T $ be a distribution on $ \nrR^n $.
The convolution seminorm function of $ T $
with index $ \Ph $ is defined by \cite{KH022a,KH023,kle024}
% \cite[Eq.\,(2.1)]{KH022a} %, \cite[p.\,1942]{KH023} and \cite{kle024}
% \cite[Eq.\,(3.2)]{KH023} and Definition~ Eq.\,(II.1.8), p.\, \cite[p.\,]{kle024}
% Equation~(2.1),       p.\,12      in \cite{KH022a}
% Equation~(3.2),       p.\,1942    in \cite{KH023}
% Equation~(I.3.8),     p.\,12,     in \cite{kle024}
% Equation~(II.1.8),    p.\,25      in \cite{kle024}
% in Section~V.2 on     p.\,131     in \cite{kle024}
% Equation~(V.2.35),    p.\,139     in \cite{kle024}
\begin{align}
    \snfConv{T}{\Ph}( x )
    &:=
    \sup_{\ph\in\Ph}\abs{ ( T * \ph )( x ) }
    &&\text{for all $ x \in \nrR^n $.}
\end{align}
The {\em diagonal cover} of a function $ f \colon \nrR^{2n} \to \nrR $
is defined by
\begin{align}
    \absDi{ f }( x, y )
    :=
    \sup_{z\in\nrR^n}\abs{ f( x + z, y + z ) }
    &&&\text{for all $ x, y \in \nrR^n $.}
\end{align}

Let $ \rmR_{\th} $ be the rotation
\begin{align}
    \rmR_{\th}( x, y )
    &=
    ( x\cos\th - y\sin\th, y\cos\th + x\sin\th )
    &&\text{for all $ x,y \in \nrR^n $, $ \th \in \nrR $.}
\end{align}
Its action on distributions is explained by
\begin{align}
    \ska{ \rmR_\th( T ) }{ \ph }
    &=
    \ska{ T }{ \ph \circ \rmR_\th }
    &&\text{for all $ \ph \in \fsTest( \nrR^{2n} ) $, $ \th \in \nrR $.}
\end{align}
The diagonal-codiagonal tensor product $ T \teproDi S $
of distributions is defined as
\begin{align}
    T \teproDi S
    &:=
    \rmR_{\pi/4}( T \tepro S )
    &&\text{for all $ T, S \in \fsDis( \nrR^n ) $,}
\end{align}
with analogous notations for completed tensor products of function spaces.

\begin{definition}
    Let $ \Ph \in \ssBnd( \fsTest( \nrR^n ) ) $.
    The codiagonal convolution seminorm functional with index $ \Ph $
    is defined as 
    \begin{align}
        \snfDiCo{ T }{ \Ph }
        &:=
        \sup_{ \ph \in \Ph }\AbsDi{ ( T * ( \sfdeo \teproDi \ph ) ) }
        &&\text{for all $ T \in \fsSmobnd(\nrR^n) \cteproDi \fsDis(\nrR^n) $.}
        \label{eq:snfDiCo}
    \end{align}
\end{definition}

\begin{lemma}
    Equation~\eqref{eq:snfDiCo} defines a Lipschitz continuous function
    for any $ T \in \fsSmobnd(\nrR^n) \cteproDi \fsDis(\nrR^n) $.
\end{lemma}

\begin{lemma}
    The ideal of seminorm functions
    $ \fsSmobnd( \nrR^n ) \cteproDi \fsDis( \nrR^n ) \to \fsCntp( \nrR^{2n} ) $
    generated by the set
    \begin{align}
        \BrmSet{ T \mto \sup_{\ph\in\Ph}\Abs{ T * \rmR_\th( \sfdeo \teproDi \ph ) } }
        { \Ph \in \ssBnd( \fsTest( \nrR^n ) ) }
    \end{align}
    is independent of $ \th \in (-\pi/2,\pi/2) $.
\end{lemma}

\begin{remark}
    Note, that $ \rmR_{-\pi/4}( \sfdeo \teproDi \ph ) = \sfdeo \tepro \ph $
    and $ \rmR_{\pi/4}( \sfdeo \teproDi \ph ) = \ph \tepro \sfdeo $.
\end{remark}